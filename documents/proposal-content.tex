\begin{center} \Large Computer Science Tripos -- Part II -- Project
Proposal\\[4mm] \LARGE Simulating Wavefronts in Real-Time using Wave
Particles\\[4mm]

\large T.~M.~Read Cutting, Downing College

Originator: Huw Bowles

13 October 2016 \end{center}

\vspace{5mm}

\textbf{Project Supervisor:} Dr R.~Mantiuk

\textbf{Director of Studies:} Dr R.~Harle

\textbf{Project Overseers:} Prof L.~Paulson  \& Dr I.~J.~Wassel

% Main document

\section*{Introduction}

Simulating water in real-time (meaning an operating speed of at least 30 frames
per second), continues to be a big challenge, consistently being used as a
watermark for the current state of real-time graphics \cite{gamingmobydick}.
With the emergence of both Virtual and Augmented Reality, the applications of
complex real-time computer graphics are only going to grow, from entertainment
all the way through to full-on education, visual aids, simulations and training.

To accurately simulate fluids such as water, full three dimensional
particle/grid-based simulation systems are required. This is something that
until very recently, was impossible to do in real-time, although advances in GPU
technology have allowed for small-scale simulations to become possible in
real-time \cite{Macklin2013}.

For large-scale systems, real-time applications use two dimensional height-map
based techniques to simulate waves. While these systems can't capture the
changes in topology caused by splashes or other phenomena where water overlaps
itself, they can operate at a scale where three-dimensional techniques still
aren't feasible.

Wave Particles are such a technique, that capture how wavefronts form and
propagate by representing them with particles on a 2D plane. \cite{Yuksel2007}.
The purpose of this project is to implement this technique, porting some of the
functionality from the CPU to the GPU for improved performance. Specifically
porting my own implementation of the rigid-body mechanics to the GPU, which
should provide some noticeable gains over the CPU-based solution used in the
Wave Particles paper.

\section*{Starting point}

I already have a theoretical understanding of the workings of the GPU, but am
unfamiliar with programming against one in practise. I will have to learn HLSL,
the shading language of Unity \cite{UnityComputeShaders}, which will then compile it to
the native shading languages of other platforms for portability.

I have read various papers surrounding water and fluid simulation, but will need
to thoroughly revise the mathematics involved.

I will start with an empty Unity project, in addition to learning the framework
as I am unfamiliar with it. The default programming language for Unity projects
is C\# \cite{UnityScripts}, which is similar to Java, a language I do have a lot
of experience with. However, it is possible to invoke C/C++ code from Unity,
something that I will investigate as it should speed up the execution of the
CPU-based simulations.

The LaTeX in the example dissertation will be used as an aid in constructing my
own one.

Finally, Huw Bowles, a professional graphics programmer, has given me advice on
which papers to read and which technologies to use, to ensure I start in the
right direction.

\section*{Resources required}

For this project I shall mainly use my own quad-core computer with an Intel i5
CPU and nVidia GTX 1070 GPU.

I will use Git as source control, backing my code up on GitHub, an external hard
drive, both my laptop and deskop, and Google Drive. Finally, the project will be
implemented within the game engine Unity, as it will allow me to focus on the
problem itself, without having to worry about setting up a lot of boilerplate.

If my computer and laptop both fail, I have money set aside in my budget to
account for that. Finally, I have an arrangement with A.~Toft (adt39) to use his
well-specced computer if all else fails.

I will keep links to all the research I do on Google Drive, in addition to
keeping a log of the research I do and the progress I make in a notebook.

\section*{Work to be done}

The project breaks down into the following sub-projects:

\begin{enumerate}

\item Create a Unity project with the required structure, which will contain all
the code that will be written for the project.

\item Implement a simple top-down 2D visualiser for the Wave Particles
themselves, which can be overlaid onto a 3D scene. This visualiser can then also
be used to view 'optimised' Wave Particles which could be implemented later.
(See extensions)

\item Implement a CPU-based solution of Wave Particles, where the user can
simply click on the 2D visualisation to simulate dipping or dragging an object
through it. Setup at least five different environments with test-obstacles for
the Wave Particles to bounce off in the simulation.

\item Render the wavefront formed by Wave Particles on a simplistic water
surface, the focus of the project IS NOT to render the water in a realistic
manner with reflections and refractions, just to simulate the behaviour of
propagating waves.

\item Port the wave-particle and fluid-object interaction modules to the GPU. It
is important that switching between the two implementations is simple to allow
for them to be compared to each other.

\end{enumerate}

\section*{Unit tests}

In theory, real-time physics simulations in Unity are deterministic. However,
they are very sensitive to the initial conditions of a system, so implementing
unit-tests in terms of actor-behaviour can be tricky. An attempt will be made to
set-up systems with consistent behaviour that can be used for unit-testing.
Failing that, what can be unit-tested are the implemented equations and
mathematics, which can be tested against expected results for certain inputs.

\section*{Success criteria}

The project will be a success if I have completed the following:

\begin{enumerate}

\item A Wave Particle simulation in Unity that runs on both the CPU and GPU with
the ability to easily compare the performance of each.

\item A fluid-object interaction system on both the CPU and GPU with the same
criteria as above.

\item Have the complete GPU-based version of the simulation run at least 30
frames-per-second with 5000 Wave Particles present.

\item A 2D top-down overlay-visualiser of the wave-particles to enable easy
visualisation of what is going on, for both explanatory and debugging purposes.

\item A basic water renderer for the wave fronts generated by the wave
particles.

\item A system to manage the Wave Particles and object placement within the
Unity editor.

\item Measurements comparing the performance of the CPU and GPU based
implementations of specific features, in addition to, data of how the number of
Wave Particles affect the performance of the simulation alongside the resulting
visual outcome.

\end{enumerate}

\section*{Possible extensions}

If I manage to achieve all of the above, I will investigate the following
possible ways of taking this project forward:

\begin{enumerate}

\item Optimise common wavefront types, by having them be represented by single
Wave Particles that encode information about the shape of the wavefront. (eg.
circular waves only need a single Wave Particle that encodes the origin of the
disturbance (from which the resulting wave can be extracted), as opposed to
having many particles arranged in a sphere. Additionally, similar
information-encoding techniques can be used for straight-lines and maybe even
certain kinds of wakes). This can also be easily visualised in the 2D
visualising tool.

\item Handle the degenerate cases where splashes should occur, as these cannot
be handled by a height-map based method such as Wave Particles. This is due the
system being unable to handle changes in surface topology. This could by done
using a simplified Smoothed Particle Hydrodynamics-like simulation, as the
transient nature of splashes means certain forces could be ignored without a
noticeable difference. The result of this can be compared with identical scenes
in a full-featured offline water simulation package, such as the one included
with the Blender animation software \cite{blenderfluid}.

\item Investigate and implement more realistic water shaders for rendering the
water. Measure how they affect the performance of the simulation in addition to
how they compare to the real physical phenomena they represent.

\item Implement the ability to view the scene in virtual-reality. Importantly,
this requires running at a framerate of at least 90 frames per second to help
prevent motion sickness that can be exacerbated by lower framerates.

\end{enumerate}

\section*{Timetable}

Planned starting date is 2016-10-20.

\begin{enumerate}

\item \textbf{Michaelmas weeks 3--4, 2016-10-20 to 2016-11-02} Thoroughly read
and understand the Wave Particles paper, reading further material as necessary.
Make notes on further reading, keeping track of references! At this point, a
first draft for the introduction to the dissertation can be constructed in
parallel to the research around the area.

\item \textbf{Michaelmas weeks 5--6, 2016-11-03 to 2016-11-16} Learn the basics
of the Unity framework by implementing a simple 2D visualiser of the Wave
Particles. Register user-input on the 2D visualiser by simply displaying a
circle where the user clicks.

\item \textbf{Michaelmas weeks 7--8, 2016-11-17 to 2016-11-30} Implement the
Wave Particle simulation on the CPU, such that the Wave Particles are simply
rendered on the 2D visualiser. Allow the user to click on the visualiser in
order to "generate" waves, by simulating a spherical object falling in the
water.

\item \textbf{Michaelmas vacation weeks 1--2, 2016-11-31 to 2016-12-14} Render
the generated wavefronts caused by the Wave Particles within the Unity renderer.
The focus here being to ensure that the surface shape of the waves is correct,
not that accurate reflections/refractions are implemented.

\item \textbf{Michaelmas vacation weeks 3--4, 2016-12-15 to 2016-12-24}
Implement fluid-object interaction on the CPU.

\item \textbf{Michaelmas vacation weeks 5--6, 2016-12-29 to 2017-01-11} Catch up
on anything I may be behind on. If everything is going according to plan by this
point, reschedule the timetable, make a decent headway on the dissertation
concerning the parts of the project already achieved, and possibly investigate
implementing any of the extensions in depth. Write a first draft of the progress
report.

\item \textbf{Lent weeks 1--2, 2017-01-19 to 2017-02-01} Finish up the progress
report, and port the relevant parts of the CPU implementation of the project to
the GPU. By the end of this period the Wave Particle simulation itself should be
implemented on the GPU.

\item \textbf{Lent weeks 3--4, 2017-02-02 to 2017-02-15} Finish port of code to
GPU. The focus here being fluid-object interaction.

\item \textbf{Lent weeks 5--6, 2017-02-16 to 2017-03-01} Run tests comparing
performance of implementation across GPU and CPU. Creating an appropriate
databank of the results and documenting the most interesting findings.

\item \textbf{Lent weeks 7--8, 2017-03-02 to 2017-03-15} Catch up on anything I
may be behind on. If everything is going according to plan, focus on getting a
first draft of the dissertation done.

\item \textbf{Lent vacation, 2017-03-16 to 2017-27-04} Focus here is entirely on
exams and the dissertation. However, if things are behind schedule, the entire
Easter vacation should provide ample time to catch-up.

\item \textbf{Easter weeks 0--2, 2017-28-04 to 2017-10-05} Proofread the
dissertation, getting feedback from people to polish things up in time for an
early submission to focus on exams!

\end{enumerate}